\documentclass{article}
\usepackage{amsmath}

\begin{document}

\section*{Exercise 7 - Conditional Probability}

The event space is:

\[
\Omega = \{ (R1_{\text{pos}}, R2_{\text{pos}}), (R1_{\text{pos}}, R2_{\text{neg}}), (R1_{\text{neg}}, R2_{\text{pos}}), (R1_{\text{neg}}, R2_{\text{neg}}) \}
\]

We are interested in the conditional probability:

\[
P(\text{R1 finds an effect} \mid \text{R2 finds an effect})
\]

This is given by:

\[
P(R1_{\text{pos}} \mid R2_{\text{pos}}) = \frac{P(R1_{\text{pos}} \cap R2_{\text{pos}})}{P(R2_{\text{pos}})}
\]
\\
\textbf{Step 1}: Compute the probability of \( R2_{\text{pos}} \)
The probability that researcher 2 finds an effect, \( P(R2_{\text{pos}}) \), is the sum of the probabilities of the elementary events where \( R2 \) finds an effect:

\[
P(R2_{\text{pos}}) = P(R1_{\text{pos}} \cap R2_{\text{pos}}) + P(R1_{\text{neg}} \cap R2_{\text{pos}})
\]
\\
\textbf{Step 2}: Compute the probability of \( R1_{\text{pos}} \cap R2_{\text{pos}} \)
This is the probability of both researchers finding an effect, \( P(R1_{\text{pos}} \cap R2_{\text{pos}}) \).\\
\\
\textbf{Step 3}: Finally, the conditional probability is:

\[
P(R1_{\text{pos}} \mid R2_{\text{pos}}) = \frac{P(R1_{\text{pos}} \cap R2_{\text{pos}})}{P(R2_{\text{pos}})} = \frac{0.04^2}{0.04 \cdot 0.04 + 0.96 \cdot 0.04} = 0.04
\]

This shows that researcher 1 finding an effect is independent from researcher 2 finding an effect.

\end{document}