\documentclass{article}
\usepackage{amsmath}  % For math formulas
\usepackage{amssymb}  % For additional math symbols
\usepackage{tikz}     % For diagrams (like Venn diagrams)
\usepackage{hyperref}
\hypersetup{
    colorlinks = true,      % Enable colored links
    linkcolor = blue,       % Internal links (e.g., table of contents)
    citecolor = blue,       % Citation links
    filecolor = blue,       % Links to local files
    urlcolor = blue         % Links to web pages
}


\begin{document}

\noindent Independence of two events $A$ and $B$ is defined as $\mathbb{P}(A \cap B) = \mathbb{P}(A) \cdot \mathbb{P}(B)$\\
\\
$A$ and $B$ are disjoint if $\mathbb{P}(A \cap B) = 0$.\\
\\
Hence, the events $A$ and $B$ can only be independent and disjoint at the same, if either $\mathbb{P}(A)$ or $\mathbb{P}(B)$ or both at the same time is $0$.\\
\\
Later: \\
Formally, this could be true for a uniformly distributed variable $X \sim U(0,1)$: \\
Event $A = \{(x=0.54345)\}$\\
Event $B = \{(x=0.886655444)\}$\\
Obviously, $A \cap B = \emptyset$ since these are different numbers and we only draw one time.
Furthermore, $\mathbb{P}(A) = \mathbb{P}(B) = 0$ since for all continuous distributions, the \href{https://stats.stackexchange.com/questions/142730/px-x-0-when-x-is-a-continuous-variable}{probability of single numbers equals zero}.

\end{document}