\documentclass[a4paper, 12pt]{article}

% Packages
\usepackage{amsmath, amssymb}
\usepackage{graphicx}
\usepackage{tikz} % If you plan to draw the tree
\usepackage{hyperref}
\usepackage{tikz}
\usetikzlibrary{positioning}

\begin{document}

\section*{Exercise 6 - Three Researchers}

\begin{enumerate}
    \item \textbf{What does the event space $\Omega$ look like?}
    
    The event space $\Omega$ for 3 researchers consists of all possible combinations of the researchers either finding a positive effect or not finding an effect. Each researcher can either find a positive effect (denoted by "pos") or not (denoted by "neg"). Thus, the event space $\Omega$ consists of $2^3 = 8$ elementary events.
    \[
    \Omega = \{(R1_{\text{pos}}, R2_{\text{pos}}, R3_{\text{pos}}), (R1_{\text{pos}}, R2_{\text{pos}}, R3_{\text{neg}}), \dots, (R1_{\text{neg}}, R2_{\text{neg}}, R3_{\text{neg}})\}
    \]
  
    \item \textbf{Which elementary events are in the set of all possible outcomes of our 3-researcher experiment and how many are there?}
    
    The total number of elementary events is $2^3 = 8$. The set of all possible outcomes is:
    \[
    \{(R1_{\text{pos}}, R2_{\text{pos}}, R3_{\text{pos}}), (R1_{\text{pos}}, R2_{\text{pos}}, R3_{\text{neg}}), (R1_{\text{pos}}, R2_{\text{neg}}, R3_{\text{pos}}), (R1_{\text{pos}}, R2_{\text{neg}}, R3_{\text{neg}}),
    \]
    \[
    (R1_{\text{neg}}, R2_{\text{pos}}, R3_{\text{pos}}), (R1_{\text{neg}}, R2_{\text{pos}}, R3_{\text{neg}}), (R1_{\text{neg}}, R2_{\text{neg}}, R3_{\text{pos}}), (R1_{\text{neg}}, R2_{\text{neg}}, R3_{\text{neg}})\}
    \]
    
    \item \textbf{Draw the corresponding binary tree for this experiment.}
    

\begin{center}
\begin{tikzpicture}[level distance=3cm, sibling distance=3.5cm, 
  edge from parent path={(\tikzparentnode) -- (\tikzchildnode)}, 
  scale=0.9, transform shape]
    % Root node (Start of the experiment)
    \node {\small Start}
    % Level 1: Researcher 1
    child {
        node {\small R1pos}
        % Level 2: Researcher 2
        child { 
            node {\small R2pos}
            % Level 3: Researcher 3
            child { node[rotate=45] {\small R3pos} }
            child { node[rotate=-45] {\small R3neg} }
        }
        child { 
            node {\small R2neg}
            child { node[rotate=45] {\small R3pos} }
            child { node[rotate=-45] {\small R3neg} }
        }
    }
    child {
        node {\small R1neg}
        child {
            node {\small R2pos}
            child { node[rotate=45] {\small R3pos} }
            child { node[rotate=-45] {\small R3neg} }
        }
        child {
            node {\small R2neg}
            child { node[rotate=45] {\small R3pos} }
            child { node[rotate=-45] {\small R3neg} }
        }
    };
\end{tikzpicture}
\end{center}
    \item \textbf{Which elementary events are in the following event: "Researcher 3 finds a positive effect"?}
    
    The event that "Researcher 3 finds a positive effect" consists of all outcomes where the third researcher's result is "pos". Thus, the event set is:
    \[
    \{(R1_{\text{pos}}, R2_{\text{pos}}, R3_{\text{pos}}), (R1_{\text{pos}}, R2_{\text{neg}}, R3_{\text{pos}}), (R1_{\text{neg}}, R2_{\text{pos}}, R3_{\text{pos}}), (R1_{\text{neg}}, R2_{\text{neg}}, R3_{\text{pos}})\}
    \]
    
    \item \textbf{Are the events "only researcher 1 finds an effect" and "only researcher 3 finds an effect" disjoint and/or independent?}
    
    The event "only researcher 1 finds an effect" is represented by the elementary event:
    \[
    (R1_{\text{pos}}, R2_{\text{neg}}, R3_{\text{neg}})
    \]
    
    The event "only researcher 3 finds an effect" is represented by the elementary event:
    \[
    (R1_{\text{neg}}, R2_{\text{neg}}, R3_{\text{pos}})
    \]
    
    These two events are \textbf{disjoint} because they cannot happen simultaneously (i.e., one requires researcher 1 to find an effect while the other requires researcher 1 not to find an effect). 
    
    Are they independent?
    
    $0 < 0.04*0.96^2 = \\
    \mathbb{P}(\text{only researcher 1 finds an effect}) \neq \\
    \mathbb{P}(\text{only researcher 1 finds an effect}|\text{only researcher 3 finds an effect}) = 0$ \\
    since this is impossible, if only researcher 3 found an effect.
    
    
    
\end{enumerate}

\end{document}